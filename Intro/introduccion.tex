
% this file is called up by thesis.tex
% content in this file will be fed into the main document
%----------------------- introduction file header -----------------------
%%%%%%%%%%%%%%%%%%%%%%%%%%%%%%%%%%%%%%%%%%%%%%%%%%%%%%%%%%%%%%%%%%%%%%%%%
%  Capítulo 1: Introducción- DEFINIR OBJETIVOS DE LA TESIS              %
%%%%%%%%%%%%%%%%%%%%%%%%%%%%%%%%%%%%%%%%%%%%%%%%%%%%%%%%%%%%%%%%%%%%%%%%%

\chapter{Introducción}\label{cap.intro}
\markboth{INTRODUCCIÓN}{INTRODUCCIÓN}


%: ----------------------- HELP: latex document organisation
% the commands below help you to subdivide and organise your thesis
%    \chapter{}       = level 1, top level
%    \section{}       = level 2
%    \subsection{}    = level 3
%    \subsubsection{} = level 4
%%%%%%%%%%%%%%%%%%%%%%%%%%%%%%%%%%%%%%%%%%%%%%%%%%%%%%%%%%%%%%%%%%%%%%%%%
%                           Presentación                                %
%%%%%%%%%%%%%%%%%%%%%%%%%%%%%%%%%%%%%%%%%%%%%%%%%%%%%%%%%%%%%%%%%%%%%%%%%

La superficie de nuestro planeta es bombardeada contínuamente por pequeñas y misteriosas partículas que viajan a través del espacio. Estas partículas son los rayos cósmicos que provienen del Sol y de fuera del Sistema Solar. Desde su descubrimiento en 1912 por el físico austriaco Victor Franz Hess\footnote{Galardonado con el Premio Nobel de Física en 1936.} y hasta el descubrimiento del antiprotón por un acelerador de partículas en 1955, la radiación cósmica ha sido el instrumento científico más importante para avanzar en el estudio de las propiedades de las partículas subatómicas\cite{rev}.\\

Los rayos cósmicos son principalmente núcleos atómicos, despojados de sus electrones por los procesos de aceleración en la interacción desde la fuente hasta la Tierra. Hay tres tipos de rayos cósmicos: los galácticos, anómalos y solares. Los \emph{rayos cósmicos galácticos} son de  energía muy alta (hasta $10^{20}\, eV$) y la mayor parte de ésta se genera a partir del nacimiento de las supernovas. Los \emph{rayos cósmicos anómalos} se originan en el medio interplanetario y se denominan así porque su composición es inusual; no sigue las abundancias naturales predichas para los diferentes isótopos\footnote{Núcleos del mismo elemento que tienen diferente número de neutrones.}\cite{cosmicray}. Hoy en día se sabe que en la atmósfera solar también se producen una gran cantidad de partículas a las que llamamos \emph{rayos cósmicos solares} y cuya energía puede ser del orden de $10^{10}$\,$eV$\footnote{Los $eV$ (electrón-volts) son unidades que miden la energía cinética adquirida por un electrón cuando es acelerado en un campo eléctrico producido por una diferencia de potencial de un volt. Ver más en \cite{mensajeros} y \cite{brunorossi}.}\cite{TNS}. El Sol, además de ser nuestra principal fuente de luz y calor, es el mayor acelerador de partículas del Sistema Solar, el cual es capaz de producir neutrones relativistas, principalmente en fulguraciones clase~X (véase el apéndice A).\\


Una fulguración solar es una explosión en el Sol que ocurre cuando la energía almacenada en campos magnéticos intensos se libera repentinamente. Esta explosión genera muchas veces una gran cantidad de partículas de energía muy alta, la mayor parte son partículas cargadas (electrones, protones y núcleos de elementos más pesados), éstas son desviadas por campos magnéticos en el sitio de aceleración, en el medio interplanetario y en el campo geomagnético. De esta forma, al llegar a la Tierra, la información física relevante se pierde. Las partículas neutras, al no ser afectadas por el campo magnético interplanetario, conservan información relevante para comprender  y calcular el espectro de los iones y su tiempo de producción.\cite{TNS}.\\
 
Los neutrones solares contienen información importante sobre los mecanismos de aceleración de iones en la atmósfera del Sol\cite{watanab}. Con base en que el tiempo de vida media de los neutrones libres es de 886 segundos, tienen que ser relativistas\footnote{$E > m_{o} c^2$ donde $m_{o}$ es la masa en reposo de la partícula y c es la velocidad de la luz en el vacío.} para que sean detectados en tierra.\\
 
El mecanismo de aceleración de los electrones ha sido estudiado mediante las observaciones con rayos X y rayos gamma (rayos $\gamma$), mientras que el de los iones aún no se comprende totalmente\cite{TNS}. La información de la aceleración de iones se transfiere a los neutrones solares y las lineas de rayos $\gamma$ que son generados por las reacciones nucleares de las partículas aceleradas con núcleos en la atmósfera solar\cite{hua1987solar}. De aquí la importancia de detectar neutrones solares.\\
 
Es necesario conocer la energía de los neutrones solares para calcular el tiempo de aceleración y saber si la aceleración de partículas es gradual o impulsiva, para ello se debe tener registro simultáneo de neutrones solares en tierra y de las emisiones rayos X y rayos $\gamma$ en el espacio\cite{TNS}.\\ 
 
 En el instituto de Geofísica de la UNAM se detectan rayos cósmicos con energías desde los 8.2 GeV\footnote{Rigidez umbral. Véase el \emph{capítulo 1}} con un instrumento que se conoce como Monitor de Neutrones\footnote{El MN instalado en C.U. es considerado uno de los detectores de neutrones más estables del mundo\cite{geo}.} (MN)\cite{geo}. El MN mide el flujo de rayos cósmicos galácticos y las variaciones debidas a las emisiones de la actividad del Sol, además de rayos cósmicos solares que se emiten cuando hay fulguraciones extremas. El MN tiene una alta sensibilidad, pero no puede conocer la dirección de arribo y energía de las partículas incidentes y no discrimina entre protones y neutrones. Por esta razón, es necesario usar un equipo nuevo especializado en la detección de neutrones solares.\\

El Telescopio de Neutrones Solares (TNS) es capaz de medir el flujo de neutrones solares, la energía primaria y diferenciar entre neutrones y protones. El nombre de telescopio se debe a que además de la capacidad mencionada anteriormente, mide la dirección de arribo de las partículas incidentes. Existe una red mundial de TNS instalados en altas montañas (Armenia, Bolivia, China, Hawai, Japón, Suiza y México). En México se instaló en la cima del volcán Sierra Negra, Puebla, en colaboración con el STELab (Solar Terrestrial Environment Laboratory) de la Universidad de Nagoya, Japón\cite{valdes}. Ha estado funcionando de manera casi continua desde julio de 2004 y ha registrado dos eventos de neutrones solares producidos en la fulguración del 7 de septiembre de 2005 y el 8 de julio de 2014\cite{Luisx,murakii}.\\

Desde 2004 hasta 2016 se tienen 13 años de registro de la detección del TNS con cuatro canales de deposición de energía. Debido al interés por observar y comprender los fenómenos y mecanismos físicos que produce el Sol, es de gran importancia hacer un análisis estadístico detallado y estudiar la estabilidad de los datos a lo largo del tiempo antes de hacer estudios de física solar. El análisis se realiza para datos de los cuatro canales de energía que registran partículas neutras, con el uso del software estadístico R.\\


Este trabajo está estructurado en 4 capítulos. En el capítulo 1 se da una breve introducción a los rayos cósmicos, en particular a los rayos cósmicos solares y se explican las principales variaciones observadas.

En el capítulo 2 se muestra la red mundial de Telescopios de Neutrones Solares; en particular las características del TNS instalado en Sierra Negra, así como el diseño, canales y capacidad de detección. También se dan las características de los datos registrados por el TNS en Sierra Negra, ya que durante los 13 de años de detección se cambió la programación del registro de datos debido a ajustes y calibración del equipo.\\ 

Con el avance de la tecnología, el proceso de adquisición de datos se ha vuelto más eficiente, esto implica gran cantidad de almacenamiento de datos. Para manipular, analizar y extraer información importante de estos datos es necesario hacer uso de algún software para programación estadística. Por esta razón, el capítulo 3 se dedica al lenguaje R, el cual se usó para llevar a cabo el análisis estadístico del presente trabajo. Se muestran los códigos programados; desde la importación y limpieza de los datos del TNS, hasta el análisis gráfico y estadístico.\\

El capitulo 4 muestra un análisis detallado de las distintas variaciones que ha presentado el flujo de partículas neutras durante 13 años para los cuatro canales de energía con anticoincidencia electrónica.\\

Finalmente se exponen las conclusiones del análisis estadístico y se muestra la señal total del detector (2004-2016). 