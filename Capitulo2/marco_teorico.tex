
%%%%%%%%%%%%%%%%%%%%%%%%%%%%%%%%%%%%%%%%%%%%%%%%%%%%%%%%%%%%%%%%%%%%%%%%%
%           Capítulo 2: MARCO TEÓRICO - REVISIÓN DE LITERATURA
%%%%%%%%%%%%%%%%%%%%%%%%%%%%%%%%%%%%%%%%%%%%%%%%%%%%%%%%%%%%%%%%%%%%%%%%%

\chapter{La red mundial de observatorios}
En este capítulo, normalmete se ponen todas las ecuaciones que se van a usar en la tesis, así ya nomás se hace rferencia a la ecuación tal o "como se vió en el capítulo 2", y esas cosas.
%inserción de codigo de Matlab
%Es conveniente sangrarlo (los de proteco dicen "indentarlo") para que no se encime con los números  de las líneas a la izquierda
\begin{lstlisting}[frame=single]
    % Declaracion de las variables simbolicas
    syms u z1 z2 z3 z4 J m M g l 
    % Matrices involucradas
    E = [J+m*l*l m*l*cos(z1);m*l*cos(z1) M+m] 
    F = [m*g*l*sin(z1);u+m*l*(z3*z3)*sin(z1)] 
    % Despeje
    V = E\F
\end{lstlisting}

\blindtext