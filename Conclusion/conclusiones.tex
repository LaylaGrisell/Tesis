\chapter*{Conclusiones}
\addcontentsline{toc}{chapter}{Conclusiones}
\markboth{CONCLUSIONES}{CONCLUSIONES}

En este trabajo, se realizó un análisis de estabilidad estadística de las partículas neutras registradas del 2004 al 2016, por  el Telescopio de Neutrones Solares instalado en Sierra Negra, Puebla. Uno de los resultados inmediatos es que se obtuvieron datos limpios, es decir, datos sin influencia de las distintas fuentes que generan errores. Esto ayudará a saber qué datos descartar y en que intervalo de tiempo se requiere hacer un análisis más detallado si se requiere extraer información física de los mismos y estudios de física solar. Además, al usar software libre (R), nos da libertad de examinar el código, de usarlo sin restricciones, de distribuirlo y modificarlo a nuestra conveniencia; de este modo, cualquiera puede acceder y entender el código utilizado para llevar a cabo el presente trabajo, de esta manera se pueden reproducir los resultados obtenidos y mejorar y adaptar los códigos a análisis estadísticos de datos que se requieran en el futuro, ahorrando tiempo en programación.\\

Con base en las medidas de centralización y dispersión que se obtuvieron por año, se pudo observar que el $50\%$ de los datos resgistrados y confiables se encuentran cerca de la media. Por otra parte, el $25\%$ de los  datos se encuentran cerca del máximo de la variación diurna y el otro 75\% más alejado . También, analizando el coeficiente de variación se pudo observar que en la mayoría de los años (excepto 2005 y 2009)  la dispersión  de los datos que corresponden a los canales S2\_A y S3\_A fue menor comparado con los otro canales, presentando una variación menor al $8\%$. Además,  los datos del canal S4\_A se dispersan más comparado con los datos de los otros canales de energía, ya que este canal detecta partículas con mayor energía; por lo tanto, las cuentas registradas son más bajas.\\

También se ha obtenido el comportamiento de la variación diurna desde el año 2004, lo cual nos da una visión de cómo se ha llevado a cabo el registro de datos a lo largo de estos años. Además, se conoce el porcentaje anual de datos  confiables, respecto al cual se concluye que el TNS funciona de forma adecuada y que, con base en la dificultad de mantener un detector a 4580 m s.n.m., el porcentaje de datos estables es muy alta. Se conocen los intervalos temporales donde los datos no son estables y/o se tiene que hacer un análisis más detallado y particular para extraer información relevante. También conocemos las variaciones por canal de energía y podemos detectar que canal tiene problemas de software, hardware o de sistema de adquisición de datos.\\

Se concluye también que la variación diurna y señal del TNS es estable. Los errores detectados en los canales se pueden atribuir a diferentes fenómenos, como son errores en la electrónica, variaciones de voltaje, fallas en el suministro de energía eléctrica y errores de programación, que pueden generar picos y/o caídas en la toma de datos y ceros.\\

Finalmente, se observó que el programa de adquisición de datos es susceptible al hardware, ya que el cambio del servidor de adquisición y almacenaje de los datos genera errores de fecha y hora, retrasos en el tiempo de 1 a varios segundos.\\

El presente análisis concluye que el TNS en Sierra Negra funciona de forma estable y los datos registrados pueden ser usados para análisis y estudios de física solar.

