
% Thesis Abstract -----------------------------------------------------


%\begin{abstractslong}    %uncommenting this line, gives a different abstract heading

\begin{abstracts}        %this creates the heading for the abstract page
\addcontentsline{toc}{chapter}{\textbf{Resumen}}
%\markboth{RESUMEN}{RESUMEN}

El Telescopio de Neutrones Solares (TNS) de México se encuentra instalado en la cima del volcán Sierra Negra, Puebla, a 4580 m s.n.m. y está operando desde el año 2004. El TNS cuenta con ocho canales de deposición de energía (E) de partículas incidentes que corresponden a $E\geq 30 \,\, MeV$, $60 \,\, MeV$, $90 \,\, MeV$ y $120 \,\, MeV$, cuatro de éstos canales corresponden a partículas cargadas y los otros cuatro a partículas neutras. Además de medir el fondo de rayos cósmicos galácticos, el TNS tiene la capacidad de detectar la energía y dirección de arribo del flujo de neutrones solares.\\

La base de datos del TNS cuenta hasta la fecha con 13 años de información sobre la detección de rayos cósmicos. En este trabajo se presenta un análisis detallado de estabilidad estadística para la serie total de datos registrados en los canales de partículas neutras desde el año 2004 hasta el año 2016, con una razón de conteo de 10 segundos. Para este análisis se utilizó el software estadístico R, donde se muestran los principales códigos ejecutados.\\

Una vez que se tienen los datos limpios y ordenados se procede con el análisis estadístico y gráfico para después pasar a la interpretación del análisis. Así, este trabajo muestra las variaciones que se han presentado en los registros de partículas neutras para conocer la calidad de los datos detectados y las distintas influencias electrónicas, eléctricas y de fenómenos de actividad atmosférica que han generado cambios en la estadística anual del TNS. Se da a conocer la señal total del detector y cómo ha variado a lo largo del tiempo. De esta manera, el análisis nos permite asegurar que se está trabajando con datos confiables para realizar los estudios básicos de física solar y poder conocer y diferenciar las afectaciones de los fenómenos puramente eléctrico-electrónicos y/o de otra índole.


\end{abstracts}
%\end{abstractlongs}


% ----------------------------------------------------------------------